\documentclass[11pt]{article}

\usepackage{times}
\usepackage[english]{babel}

% -----------------------------------------------
% especially use this for you code
% -----------------------------------------------

\usepackage{courier}
\usepackage{listings}
\usepackage{color}
\usepackage{tabularx}
\usepackage{graphicx}

\definecolor{Gray}{gray}{0.95}

\definecolor{mygreen}{rgb}{0,0.6,0}
\definecolor{mygray}{rgb}{0.5,0.5,0.5}
\definecolor{mymauve}{rgb}{0.58,0,0.82}

\lstset{language=C++,
	basicstyle = \normalsize\ttfamily,   % the size and fonts that are used
	tabsize = 2,                    % sets default tabsize
	breaklines = true,              % sets automatic line breaking
	keywordstyle=\color{blue}\ttfamily,
	stringstyle=\color{red}\ttfamily,
	commentstyle=\color{mygreen}\ttfamily,
	numbers=left,
	keepspaces=true,
	showspaces=false,
	showstringspaces=false,
}

\begin{document}

\title{Programming in C/C++ \\
       Exercises set six: multi-threading 1
}
\date{\today}
\author{Christiaan Steenkist \\
Jaime Betancor Valado \\
Remco Bos \\
}

\maketitle
\section*{Exercise 38, basic multithreading with recursive program}
We were tasked to design a program that recursively visits all elements of a starting directory and computes the total size of its regular files.

\subsection*{Code listings}
\lstinputlisting[caption = main.ih]{src/a38/main.ih}
\lstinputlisting[caption = main.cc]{src/a38/main.cc}
\lstinputlisting[caption = openfolder.cc]{src/a38/openfolder.cc}
\lstinputlisting[caption = timeprocess.cc]{src/a38/timeprocess.cc}

\section*{Exercise 39, using chrono/clock facilities }
We were tasked to display the time at the beginning and end of a program.

\subsection*{Output}
\lstinputlisting[caption = output]{src/a39/output.txt}

\subsection*{Code listings}
\lstinputlisting[caption = main.cc]{src/a39/main.cc}

\section*{Exercise 40, thread-safe queue}
A proxy and a lot of lock guards attempt to make this queue thread-safe.

\subsection*{Code listings}
\lstinputlisting[caption = safequeue.ih]{src/a40/main.ih}
\lstinputlisting[caption = safequeue.h]{src/a40/safequeue.h}
\lstinputlisting[caption = back.cc]{src/a40/back.cc}
\lstinputlisting[caption = empty.cc]{src/a40/empty.cc}
\lstinputlisting[caption = front.cc]{src/a40/front.cc}
\lstinputlisting[caption = \texttt{opeartor\=1.cc}]{src/a40/operator=1.cc}
\lstinputlisting[caption = \texttt{operator\=2.cc}]{src/a40/operator=2.cc}
\lstinputlisting[caption = pop.cc]{src/a40/pop.cc}
\lstinputlisting[caption = promotor.cc]{src/a40/promotion.cc}
\lstinputlisting[caption = proxyconstr.cc]{src/a40/proxyconstr.cc}
\lstinputlisting[caption = push.cc]{src/a40/push.cc}
\lstinputlisting[caption = queueconstr.cc]{src/a40/queueconstr.cc}

\section*{Exercise 42, establish connection between a parent and child process}
We were tasked to pass output from a child process to the parent process by using fork and exec.

\subsection*{Code listings}
\lstinputlisting[caption = main.cc]{src/a42/main.cc}
\lstinputlisting[caption = main.h]{src/a42/main.h}
\lstinputlisting[caption = usechildoutput.cc]{src/a42/usechildoutput.cc}

\section*{Exercise 43: design a simple multi-thread program}
We were tasked to make a program with threads counting vowels, hexadecimals, digits and punctuation character in a file passed to the program.

\subsection*{Code listings}
\lstinputlisting[caption = task.h]{src/a43/task.h}
\lstinputlisting[caption = task.ih]{src/a43/task.ih}
\lstinputlisting[caption = main.cc]{src/a43/main.cc}
\lstinputlisting[caption = task.cc]{src/a43/task.cc}
\lstinputlisting[caption = countdigit.cc]{src/a43/countdigit.cc}
\lstinputlisting[caption = counthexdec.cc]{src/a43/counthexdec.cc}
\lstinputlisting[caption = countpunctchar.cc]{src/a43/countpunctchar.cc}
\lstinputlisting[caption = countvowelcc]{src/a43/countvowel.cc}

\end{document}
